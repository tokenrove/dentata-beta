%
% devintro.tex
% 
% Created: Sat Jun 23 00:06:03 2001 by tek@wiw.org
% Revised: Sun Jun 24 07:39:26 2001 by tek@wiw.org
% Copyright 2001 Julian E. C. Squires (tek@wiw.org)
% 
\def\FileCreated{Sat Jun 23 00:06:03 2001}
\def\FileRevised{Sun Jun 24 07:39:26 2001}

\section{The dentata Source}

\subsection{File Names}

During the MSDOS port, all the filenames were abbreviated to
8.3 style, so they may not be immediately parsable at first
glance.

Most of the names are a combination of a five-letter identifier
(for example, ``image'' or ``rastr'') and a three-letter suffix.
Table \ref{tab:tlsuffixes} shows the various three letter suffixes
in use.

\begin{table}[hpt]
\caption{Three Letter Suffixes Used in Source Naming in dentata}
\label{tab:tlsuffixes}
\begin{tabular}{|c|c|}
\hline
Suffix & Meaning \\
\hline\hline
gen & Generic \\
com & Common \\
nul & Null (skeleton) \\
std & Standard C Library specific \\
\hline\hline
arm & ARM/Thumb specific \\
mip & MIPS specific \\
sh3 & SuperH 3 specific \\
sh4 & SuperH 4 specific \\
x86 & Intel x86 specific \\
\hline\hline
dc & Dreamcast specific \\
dos & MSDOS specific \\
gba & Gameboy Advance specific \\
psx & Playstation specific \\
ps2 & Playstation 2 specific \\
sat & Sega Saturn specific \\
unx & UNIX/POSIX specific \\
w32 & Windows 95/98/NT specific \\
\hline\hline
dxn & DirectX n specific \\
vgl & SVGAlib specific \\
x11 & X Window System specific \\
oss & OSS (Open Sound System) specific \\
\hline
\end{tabular}
\end{table}

\subsection{Coding Style}

The C coding style used within dentata is basically four-space
K\&R. The best way of determining the general coding style is to
look at the code itself. There are a number of much more strict
rules in place on the naming of identifiers, however. These
rules are summarized in table \ref{tab:idconvsummary}.

In general, anything visible to the user is prefixed with {\tt d\_},
even defined constants.

Module functions which are visible to the user are always prefixed with
{\tt d\_}, with the module name in the center, followed by the function
name, seperated by an underscore. No module data should be explicitly
visible to the user.

Internal routines which need to be shared between files should be similar
in form to the user-level module routines, save for prefixing with an
underscore to indicate that it is an internal routine.
Internal routines which are only used by one file should be declared
static and should not have any prefixes, that is, just {\tt
function}().

Structures are almost always typedef'd into types. Defined types (except
for the basic types listed in table \ref{tab:denttypes}) have the
suffix {\tt \_t}, like {\tt d\_type\_t}. The convention in defining
structures is demonstrated in the following code snippet. (note the
prefixes and suffixes used)

\begin{verbatim}
typedef struct d_type_s {
    /* members */
} d_type_t;
\end{verbatim}

Types visible by users or shared between modules are in the
form {\tt d\_type\_t}. Types used by a single module (which may
span multiple files) should be of the form {\tt type\_t}.

Defined constants are all uppercase, in the form {\tt D\_MODULE\_CONSTANT}
when visible outside a module, or, less commonly, {\tt D\_CONSTANT} when
the constant applies to multiple modules. Internal constants do not have
any prefix, and so are more like: {\tt CONSTANT}.

\begin{table}
\caption{Identifier Conventions Summary}
\label{tab:idconvsummary}
\begin{tabular}{|c|c|}
\hline
Type of Identifier & Example \\
\hline\hline
User-visible module functions & {\tt d\_module\_function}() \\
Internal routines (shared) & {\tt \_d\_module\_function}() or {\tt
\_d\_function}() \\
Internal routines (static) & {\tt function}() \\
User-visible or shared types & {\tt d\_type\_t} \\
Module-internal types & {\tt type\_t} \\
Module defined constant & {\tt D\_MODULE\_CONSTANT} \\
General defined constant & {\tt D\_CONSTANT} \\
Internal defined constant & {\tt CONSTANT} \\
\hline
\end{tabular}
\end{table}

\subsection{Important Practices}

\begin{description}
\item [Keeping Documentation Up-to-Date]
Keep in mind that developers depend on this manual being up to date, and a
large portion of it (the function reference) is generated from
comments. {\bf Every} modification to a public (API) function should
cause you to think: ``Do I need to update the header comments due to
this change?''. Adding a module or adding new public functions to an
existing module without updating the headers is unthinkable.

\item [Using CVS]
Read the guide to CVS which is on the rhombus website. Learning to
use CVS effectively is something which can make a huge difference in
productivity, especially in cases where hardware dies and data is
lost, or when bugs are introduced to previously stable code. Commit
your changes frequently.

\end{description}
