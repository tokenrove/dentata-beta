%
% devintro.tex
% 
% Created: Sat Jun 23 00:06:03 2001 by tek@wiw.org
% Revised: Sat Jun 23 01:16:20 2001 by tek@wiw.org
% Copyright 2001 Julian E. C. Squires (tek@wiw.org)
% 
\def\FileCreated{Sat Jun 23 00:06:03 2001}
\def\FileRevised{Sat Jun 23 01:16:20 2001}

\section{The dentata Source}

\subsection{File Names}

During the MSDOS port, all the filenames were abbreviated to
8.3 style, so they may not be immediately parsable at first
glance.

Most of the names are a combination of a five-letter identifier
(for example, ``image'' or ``rastr'') and a three-letter suffix.
Table \ref{tab:tlsuffixes} shows the various three letter suffixes
in use.

\begin{table}[hpt]
\caption{Three Letter Suffixes Used in Source Naming in dentata}
\label{tab:tlsuffixes}
\begin{tabular}{|c|c|}
\hline
Suffix & Meaning \\
\hline\hline
gen & Generic \\
com & Common \\
nul & Null (skeleton) \\
std & Standard C Library specific \\
\hline\hline
arm & ARM/Thumb specific \\
mip & MIPS specific \\
sh3 & SuperH 3 specific \\
sh4 & SuperH 4 specific \\
x86 & Intel x86 specific \\
\hline\hline
dc & Dreamcast specific \\
dos & MSDOS specific \\
gba & Gameboy Advance specific \\
lnx & Linux specific \\
psx & Playstation specific \\
ps2 & Playstation 2 specific \\
sat & Sega Saturn specific \\
unx & UNIX/POSIX specific \\
w32 & Windows 95/98/NT specific \\
\hline\hline
dx<n> & DirectX <n> specific \\
x11 & X Window System specific \\
\hline
\end{tabular}

\end{table}