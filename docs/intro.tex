%
% intro.tex
% 
% Created: Fri Jun 22 17:12:01 2001 by tek@wiw.org
% Revised: Sun Jun 24 04:14:12 2001 by tek@wiw.org
% Copyright 2001 Julian E. C. Squires (tek@wiw.org)
% 
\def\FileCreated{Fri Jun 22 17:12:01 2001}
\def\FileRevised{Sun Jun 24 04:14:12 2001}

Dentata is a library of modules for game developers. Its principle
aim is to provide a comprehensive set of services to allow game
development, without hindering portability. Secondarily, dentata
provides highly optimized implementations on many platforms, with
the intent of drastically reducing game development time.

\section{Where to Start}
\label{sec:wheretostart}

This chapter is written with the intent of providing 70\% of the
information required for a new developer to produce their first
demo with dentata. The remaining 30\% should be covered by the
following chapters.

\subsection{Basic Types}

First of all, both to facillitate game development and to
enhance portability, dentata uses its own set of types, each
with a specific size, independant of platform. Table \ref{tab:denttypes}
enumerates the types of primary importance.

\begin{table}[hbt]
\caption{Basic types in dentata}
\label{tab:denttypes}
\begin{tabular}{|c|c|c|}
\hline
Type name & Size & Range of values \\
\hline\hline
{\tt bool} & 1 bit & {\tt true}, {\tt success} or {\tt false}, {\tt failure} \\
{\tt byte} & 8 bits & 0 to 255 \\
{\tt word} & 16 bits & 0 to 65535 \\
{\tt dword} & 32 bits & 0 to 4294967296 \\
\hline
\end{tabular}
\end{table}

Most modules will have a {\tt new}() function which initializes
internal state, and which returns a {\tt bool} to indicate whether it
was successful.

Generally, all the routines are in the form of {\tt d\_}{\it
module}{\tt \_}{\it function}.

\subsection{Getting a Display}

Given that the primary mode of communication from a game to its
user is via graphics, the first priority of a developer is to
enable a graphical display. Within dentata, the raster module is
responsible for providing the interface to the display.

The first step is to simply initialize the raster device. The following
snippet of code does this, aborts if the initialization failed, and then
cleanly shuts down the raster device.

\begin{verbatim}
    bool status;

    /* setup the raster device */
    status = d_raster_new();
    /* if the call failed, bomb out */
    if(status != success)
        d_error_fatal("Unable to initialize the raster"
                      " device.\n");
    /* shut down the raster device */
    d_raster_delete();
\end{verbatim}

Assuming that worked acceptably, the next step is to find out what
graphics modes are available to us. The {\tt d\_raster\_getmodes}()
routine is used for this purpose. The following routine would go in
between the call to {\tt d\_error\_fatal}() and the call to {\tt
d\_raster\_delete}() in the code above.

\begin{verbatim}
    int nmodes;
    d_rasterdescription_t *modes;

    /* get a list of supported modes */
    modes = d_raster_getmodes(&nmodes);
\end{verbatim}

One might now find it of interest to look at the source code for
the {\tt lstmodes} tool included in the dentata distribution.

The final step is to actually set a mode. One of the problems of
writing portable game code is that raster capabilities vary wildly
between systems. Therefore, developers are expected to write code
which intelligently chooses the mode to use, and from that, adapts
image data and algorithms to fit this mode.\footnote{Luckily, the rhombus
snippets library includes a snippet called {\tt smarthaggle}, which does
most of the legwork for you.}

A very simplistic way of choosing a mode is as follows:
\begin{verbatim}
    int i, desiredwidth = 320, desiredheight = 200,
        desiredbpp = 16;

    /* assume the previous snippet is here */

    for(i = 0; i < nmodes; i++) {
        /* the only particularly important aspects of
         * a mode are its width, height, and bits per
         * pixel. */
        if(modes[i].w == desiredwidth &&
           modes[i].h == desiredheight &&
           modes[i].bpp == desiredbpp) {
             /* attempt to set the mode */
             status = d_raster_setmode(modes[i]);
             /* if we succeeded, break out of the loop
              * early. */
             if(status == success)
                 break;
        }
    }
    /* did we exit the loop normally? */
    if(i >= nmodes)
        d_error_fatal("Unable to setup a suitable raster"
                      " mode (desired: %dx%dx%d)\n",
                      desiredwidth, desiredheight,
                      desiredbpp);
\end{verbatim}

Before we go any further into display, we need to take a look at
dealing with user input.

\subsection{Trivial Event Handling}

The event handling system in dentata attempts to support as many input
devices as could conceivably be used in a game. Principally, game
developers are interested in digital inputs, that is, inputs which are
exclusively either on or off. The event module also deals with analog
inputs, but we will only concern ourselves with digital in this
section.

The first step is, as before, initializing the module, but in this
case, {\tt d\_event\_new}() takes an argument, {\tt byte \it mask}.
The header file {\tt dentata/event.h} defines a number of event masks,
each for a different class of devices supported. In this case, we'd
liketo initialize event handling for keyboards and controllers, so
the following code is used.

\begin{verbatim}
    byte mask, desiredmask;

    desiredmask = D_EVENT_KEYBOARD|D_EVENT_CONTROLLER;
    mask = d_event_new(desiredmask);
    if(mask == 0)
        d_error_fatal("Unable to initialize any input "
                      "devices.\n");

     /* program body continues here... */

     d_event_delete();
\end{verbatim}

Note that we don't compare the return value of {\tt d\_event\_new}()
with the desired mask, as we want support for keyboard {\bf or}
controller, but we do not need both.

The next step in adding digital input handling to your program is
to setup {\bf mappings}. Mappings are associations between input
events and arbitrary handles of the programmer's choosing. This
allows one to easily support multiple input devices, aliases for
events, and redefinable keys.

A typical use of mappings is demonstrated in the next snippet.

\begin{verbatim}
/* This would typically be put in a header file for the
 * project. */
enum { EV_QUIT = 0, EV_LEFT, EV_RIGHT, EV_UP, EV_DOWN };

/* ... */
    /* skip to just after the event module has been
     * initialized */
    d_event_map(EV_QUIT, D_KBD_ESCAPE);
    d_event_map(EV_QUIT, D_KBD_Q);
    d_event_map(EV_QUIT, D_CTRLR_START);

    d_event_map(EV_LEFT, D_KBD_CURSORLEFT);
    d_event_map(EV_LEFT, D_CTRLR_LEFT);
    /* likewise mappings for EV_RIGHT, EV_UP, EV_DOWN
     * would be here */
\end{verbatim}

Now, the remaining question is: how does one actually check for events?
The following code snippet shows a skeleton of a typical game loop.

\begin{verbatim}
    while(1) {
        d_event_update();
        if(d_event_ispressed(EV_QUIT))
            break;
        /* game content update would be here */
        /* graphics and audio updates would be here */
    }
\end{verbatim}

One should note that while event handling will work fine without regular
calls to the {\tt d\_event\_update}() function on many platforms, to write
portable code one {\bf must} call it every time the program needs fresh
event data.

\subsection{Accessing the Framebuffer}

By this point, one should be able to write a program which sets up a
graphics mode and waits for the user to signal for them to quit.
However, the part that is probably the most interesting to most
readers is how to actually get things happening on the screen.

The raster module in dentata implicitly provides a double-buffer. How
that double-buffer is implemented depends on the platform, but direct
access can be obtained by calling {\tt d\_raster\_getgfxptr}(). Usually,
though, one wants dentata to handle all the issues of figuring out how
to pack pixels with this mode and other such details. While there are a
number of ways of doing this, the kosher, 100\% approved way is to setup
a dummy image, as demonstrated in the following piece of code.

\begin{verbatim}
    d_image_t *raster;

    /* raster init code goes here */

    /* create a new image of the same size as the raster
     * display */
    raster = d_image_new(d_raster_getcurrentmode());

    /* set the image to directly access the raster's
     * double buffer */
    d_image_setdataptr(raster, d_raster_getgfxptr(),
                       false);

    /* use the image as you wish */

    /* free the image */
    d_raster_delete();
    d_image_delete(raster);
\end{verbatim}

Now one can use dentata's image module to operate on the raster
double buffer. Only one thing remains -- how do we swap the double
buffer onto the display? The following snippet illustrates the use
of {\tt d\_raster\_update}(), which is the function to update the
screen with the contents of the double buffer.

\begin{verbatim}
    int i = 0;

    while(1) {
       d_event_update();
       if(d_event_ispressed(EV_QUIT))
           break;

       d_image_wipe(raster, i++, 255);
       d_raster_update();
    }
\end{verbatim}

\subsection{Frame Timing}

Dentata includes a few useful timer related functions, but the most
essential for a game programmer is the code to keep frames on a
constant time. The most basic form of this is to use the {\tt
d\_time\_startcount}() and {\tt d\_time\_endcount}() functions with a
static framerate. This is shown below.

\begin{verbatim}
    d_timehandle_t *th = NULL;

    while(1) {
        th = d_time_startcount(framespersecond, false);
        /* game is updated, raster is redrawn, etc. */
        d_time_endcount(th);
        th = NULL;
    }

    /* just in case the loop terminated prematurely.
     * avoids a potential memory leak. */
    if(th) d_time_endcount(th);
\end{verbatim}

Most of the usage of these functions should be fairly obvious from
the above snippet (if not, see section \ref{sec:time}), except for
the second argument to {\tt d\_time\_startcount}(). When this is
{\tt true}, {\tt d\_time\_endcount}() will return an {\tt int} which
specifies how many microseconds were left for the current frame.
This value can be positive or negative, and is intended to allow
developers to adjust their framerate on the fly, to cope with systems
with variable performance.\footnote{The rhombus snippets contain this
code, under {\tt autofps}.}

\section{General Game Structure}

\subsection{Initialization}

\subsection{Game Loop}
\subsection{Shutdown}

\section{Notes on Portability}

Avoid using the standard C library. Dentata is targetted at many
platforms which lack an implementation of the standard C library, and
so includes more game-oriented replacements for a large degree of the
functionality. For example, the file module. By default, on a PC, it
uses the standard C library to perform operations on files. However,
on a gameboy advance, it acts as a symbol table for addresses in
ROM. In fact, there is even a module to allow one to use data packed
into a single file on the PC, in a fasion similar to the DAT or
WAD systems used by many commercial games.

One cannot depend on {\tt argc} or {\tt argv} being usable by the
user, as well, one cannot depend on the user being able to see
debugging messages or the like, which is why no direct {\tt printf}()
replacement exists. A combination of the error module and the font
module provide for the needs of programs to output debugging
information.

Do not assume things like images and samples go into the same place
in memory as everything else. Many console systems do not allow this,
and so the modules are tuned appropriately on those platforms.
Also, remember that video and audio memory may be quite limited,
and while dentata will attempt to intelligently cache the data which
has been least recently used in order to free enough space, this
is not a license to keep unused images lying around in memory.\footnote{A
forthcoming rhombus snippet will demonstrate good practice with
respect to this issue in a game divided into different parts (with
little shared data).}
